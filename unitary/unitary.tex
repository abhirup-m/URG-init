\documentclass[12pt]{report}
\usepackage{common}
\begin{document}
The goal is to obtain the unitary transformation. In the half-filled subspace for the Hubbard dimer,
\beq
\ham = \bordermatrix{~ & \ket{\ua,\da} & \ket{\ua\da,0} & \ket{\da,\ua} & \ket{0,\ua\da} \cr
	& 0 & t & 0 & -t \cr \\
	& t & U & -t & 0 \cr \\
	& 0 & -t & 0 & t\cr \\
	& -t & 0 & t & U \cr}
\eeq
I have dropped the part of the Hamiltonian involving \il{\ket{\ua,\ua}} and \il{\ket{\da,\da}} because they are already decoupled and do not change under the RG. Notice that \il{\ham} can be written as 
\beq
	\ham = \bordermatrix{~ & \ket{n_{1\ua} = 1} & \ket{n_{1\ua}=0} \cr
		& a & b \cr\\
		& b & a \cr}
\eeq
Applying RG on this matrix,
\begin{gather}
H_e = H_h = a \\
T = b\\
\eta^\dagger = \fr{1}{E - H_e}c^\dagger_{1\ua} T = \fr{1}{E - a}c^\dagger_{1\ua}b\\
\implies \eta  = b^\dagger c_{1\ua}\fr{1}{E - a}
\end{gather}
From properties of \il{\eta},
\begin{gather}
\hat n_{1\ua} = \eta^\dagger \eta = \fr{1}{E-a}c^\dagger_{1\ua}c_{1\ua}bb^\dagger\fr{1}{E-a} \\
\implies(E-a)^2 \hat n_{1\ua} = \hat n_{1\ua}b^2
\end{gather}
I used \il{b = -t\sigma_x \implies b^\dagger = b}. The two solutions for \il{E} are
\begin{gather}
E -a = \pm b \\
\implies E_\pm = a \pm b\\
\implies \ham_\text{rotated} = \bordermatrix{~ & \ket{n_{1\ua} = 1} & \ket{n_{1\ua}=0} \cr
		& a-b & 0 \cr
		& 0 & a+b \cr}
		=\begin{pmatrix}
		  0 & 2t & 0 & 0 \\
		  2t & U & 0 & 0 \\
		  0 & 0 & 0 & 0 \\
		  0 & 0 & 0 & U
	  \end{pmatrix}
\end{gather}
For this step, the unitary is
\beq
U_1 = \fr{1}{\sqrt 2} \bordermatrix{~ & \ket{n_{1\ua} = 1} & \ket{n_{1\ua}=0} \cr
				      & -1 & 1 \cr\\
				      & 1 & 1 \cr}
	= \fr{1}{\sqrt 2} \begin{pmatrix} \\ & -\mathbb{I}_{2\times 2} & \mathbb{I}_{2\times 2} &\\\\\\
	& \mathbb{I}_{2\times 2} & \mathbb{I}_{2\times 2} & \\
	&&&
	\end{pmatrix}
\eeq
Taking a look at \il{\ham_\text{rotated}}, the lower block is diagonal. So, take the upper block as the new Hamiltonian,
\begin{gather}
\ham = \bordermatrix{~ &  \ket{0,\da} & \ket{\da,0}\cr
	     & 0 & 2t \cr
	     & 2t & U \cr}\\
H_e = 0, H_h = U, T = 2t \\
\eta^\dagger = \fr{1}{E - H_e} c_{1\da}^\dagger T = \fr{2t}{E} c_{1\da}^\dagger \\
\eta = \fr{1}{E - H_h} T^\dagger c_{1\da} = \fr{2t}{E - U} c_{1\da}\\
\hat n_{1\da} = \eta^\dagger \eta = \fr{4t^2}{E(E-U)}\hat n_{1\da}\\
\implies E(E-U)=4t^2 \implies E = \fr{U\pm\Delta}{2}
\end{gather}
Therefore,
\begin{gather}
	\ham_{rotated} = \begin{pmatrix}\fr{U-\Delta}{2} & 0 \\ 0 & \fr{U+\Delta}{2} \end{pmatrix}\\
	\mathcal{U} = \begin{pmatrix} \fr{4t}{U-\Delta} & \fr{4t}{U+\Delta} \\ 1 &1 \end{pmatrix}
	\end{gather}
\il{N_\pm^2 = \Delta\rr{\Delta\pm U}}. Since this unitary acts only on the upper block, the complete unitary for this stage is
\beq
		U_2 = \begin{pmatrix} \mathcal{U} & 0_{2\times 2} \\
		0_{2\times 2} & \mathbb{I}_{2\times 2}\end{pmatrix}
\eeq
The total unitary for the entire diagonalization process is
\beq
U = U_1 \times U_2 = \fr{1}{\sqrt 2}\begin{pmatrix}
	-\mathcal{U} & 1 \\
	\mathcal{U} & 1
	\end{pmatrix}
\eeq
To check whether these are correct, we can compute the eigenstates. The eigenstates of the unitarily rotated Hamiltonian are just the fermionic degrees of freedom \il{\ket{n_{i\sigma}}}. The eigenstates of the bare Hamiltonian are hence obtained by rotating these states:
\beq
			\overline {\ket{\psi_1}}=U\ket{\ua,\da} = \fr{1}{\sqrt 2}\begin{pmatrix} -\mathcal{U} & 1 \\& \\ 
				\mathcal{U} & 1 \end{pmatrix} \begin{bmatrix} 1 \\ 0 \\ 0 \\ 0 \end{bmatrix} &= \fr{1}{\sqrt 2}\begin{bmatrix} -\fr{4t}{U-\Delta} \\ -1 \\ \fr{4t}{U-\Delta} \\ 1 \end{bmatrix} \sim \fr{1}{\sqrt 2}\cc{\fr{4t}{U-\Delta}\rr{\ket{\ua,\da} - \ket{\da,\ua}}+\rr{\ket{\ua\da,0}-\ket{0,\da\ua}}}\\
				\overline {\ket{\psi_2}}=U\ket{\ua,\da} &=  \fr{1}{\sqrt 2}\begin{bmatrix} -\fr{4t}{U+\Delta} \\ -1 \\ \fr{4t}{U+\Delta} \\ 1 \end{bmatrix} \sim \fr{1}{\sqrt 2}\cc{\fr{4t}{U+\Delta}\rr{\ket{\ua,\da} - \ket{\da,\ua}}+\rr{\ket{\ua\da,0}-\ket{0,\da\ua}}} \\
				\overline {\ket{\psi_3}}=U\ket{\ua,\da} &=  \begin{bmatrix} 1 \\ 0 \\ 1 \\ 0 \end{bmatrix} \sim \fr{1}{\sqrt 2}\rr{\ket{\ua,\da} + \ket{\da,\ua}}\\
				\overline {\ket{\psi_4}}=U\ket{\ua,\da} &= \begin{bmatrix} 0 \\ 1 \\ 0 \\ 1 \end{bmatrix} \sim \fr{1}{\sqrt 2}\cc{\ket{\ua\da,0} + \ket{0,\ua\da}}
\eeq
Now that we have the unitary transformaiton, the contention is that the following is the correct effective Hamiltonian:
\beq
\overline \ham = U\hat n_{2\ua}\hat n_{2\da} + \fr{U-\Delta}{2}\hat n_{1\ua}\hat n_{2\da}+ \fr{U+\Delta}{2}\hat n_{1\ua}\hat n_{1\da}
\eeq
\subsubsection*{Matching eigenvalues}
To check that this gives the correct eigenvalues, I operate this on the states which should be its eigenstates, that is, the decoupled degrees of freedom \il{\ket{n_{i\sigma}}}:
\beq
\ol \ham \ket{\ua,\da}&=\begin{pmatrix} \fr{U-\Delta}{2} & 0 & 0 & 0 \\ 0 & \fr{U+\Delta}{2} & 0 & 0 \\ 0 & 0 & 0 & 0 \\ 0 & 0 & 0 & U \end{pmatrix}\begin{bmatrix} 1 \\ 0 \\ 0 \\ 0 \end{bmatrix} = \fr{U-\Delta}{2}\\
\ol \ham \ket{\ua\da,0}&=\begin{pmatrix} \fr{U-\Delta}{2} & 0 & 0 & 0 \\ 0 & \fr{U+\Delta}{2} & 0 & 0 \\ 0 & 0 & 0 & 0 \\ 0 & 0 & 0 & U \end{pmatrix}\begin{bmatrix} 0 \\ 1 \\ 0 \\ 0 \end{bmatrix} = \fr{U+\Delta}{2}\\
\ol \ham \ket{\da,\ua}&=\begin{pmatrix} \fr{U-\Delta}{2} & 0 & 0 & 0 \\ 0 & \fr{U+\Delta}{2} & 0 & 0 \\ 0 & 0 & 0 & 0 \\ 0 & 0 & 0 & U \end{pmatrix}\begin{bmatrix} 0 \\ 0 \\ 1 \\ 0 \end{bmatrix} = 0\\
\ol \ham \ket{0,\ua\da}&=\begin{pmatrix} \fr{U-\Delta}{2} & 0 & 0 & 0 \\ 0 & \fr{U+\Delta}{2} & 0 & 0 \\ 0 & 0 & 0 & 0 \\ 0 & 0 & 0 & U \end{pmatrix}\begin{bmatrix} 0 \\ 0 \\ 0 \\ 1 \end{bmatrix} = U\\
\eeq
Next is a proof that \il{\ol \ham} is unitarily linked with the bare Hamiltonian by the same unitary transformation:
\beq
U \overline \ham U^{-1} = U	\begin{pmatrix} \fr{U-\Delta}{2} & 0 & 0 & 0 \\ 0 & \fr{U+\Delta}{2} & 0 & 0 \\ 0 & 0 & 0 & 0 \\ 0 & 0 & 0 & U \end{pmatrix} U^{-1} = \begin{pmatrix}
	0 & t & 0 & -t  \\
	t & U & -t & 0  \\
	0 & -t & 0 & t \\
-t & 0 & t & U \end{pmatrix} = \ham
\eeq
This proves that \il{\overline \ham} shares the symmetries of \il{\ham}.\\\\

\subsubsection*{Rotated spin-inversion operator}
The spin-inversion operator in the original basis is
\beq
T = \begin{pmatrix} 0 & 0 & 1 & 0 \\ 0 & -1 & 0 & 0\\ 1 & 0 & 0 & 0\\0& 0 & 0 & -1 \end{pmatrix} = \begin{pmatrix} a & a + \mathbb{I} \\ a + \mathbb{I} & a \end{pmatrix}
\eeq
where \il{a = \begin{pmatrix} 0 & 0 \\ 0 & -1 \end{pmatrix}}.
To see the nature of the rotated spin-inversion operator,
\beq
	\ol{T} = U^{-1} T U = \begin{pmatrix} -\mathbb{I} & 0 \\ 0 & \sigma_z \end{pmatrix}
\eeq
To see that \il{\ol T} commutes with \il{\ol H},
\beq
\bar T \bar H &= \begin{pmatrix}-\mathbb{I} & 0 \\ 0 & \sigma_z \end{pmatrix}\begin{pmatrix} \fr{U-\Delta}{2} & 0 & 0 & 0 \\ 0 & \fr{U+\Delta}{2} & 0 & 0 \\ 0 & 0 & 0 & 0 \\ 0 & 0 & 0 & U \end{pmatrix}  = \begin{pmatrix} -\fr{U-\Delta}{2} & 0 & 0 & 0 \\ 0 & -\fr{U+\Delta}{2} & 0 & 0 \\ 0 & 0 & 0 & 0 \\ 0 & 0 & 0 & -U \end{pmatrix}\\
\bar H \bar T &= \begin{pmatrix} \fr{U-\Delta}{2} & 0 & 0 & 0 \\ 0 & \fr{U+\Delta}{2} & 0 & 0 \\ 0 & 0 & 0 & 0 \\ 0 & 0 & 0 & U \end{pmatrix}  \begin{pmatrix}-\mathbb{I} & 0 \\ 0 & \sigma_z \end{pmatrix} = \begin{pmatrix} -\fr{U-\Delta}{2} & 0 & 0 & 0 \\ 0 & -\fr{U+\Delta}{2} & 0 & 0 \\ 0 & 0 & 0 & 0 \\ 0 & 0 & 0 & -U \end{pmatrix}
\eeq

\subsection*{Entanglement}
\beq
\ket{GS} = \fr{1}{\sqrt{2\rr{1+\alpha^{-2}}}}\qq{\ket{\ua,\da}-\ket{\da,\ua} +\alpha^{-1}\rr{\ket{0,\ua\da} - \ket{\ua\da,0}}}
\eeq
To determine the entanglement between the two states, I compute the von Neumann entropy of the reduced density matrix of the site 1 obtained by tracing out site 2.
\beq
\rho &= \ket{GS}\bra{GS}\\
\rho_1 &= \sum_{\ket{x}_2}\bra{x}\rho\ket{x}
\eeq
where the sum is over all the configurations of the second site: \il{\{\ket{x}\} = \{\ket{0},\ket{\ua},\ket{\da},\ket{\ua\da}\}}
\beq
\bra{0}_2 \rho \ket{0}_2 &= \fr{\alpha^{-2}}{2(1+\alpha^{-2})}\ket{\ua\da}\bra{\ua\da}\\
\bra{\ua}_2 \rho \ket{\ua}_2 &= \fr{1}{2(1+\alpha^{-2})}\ket{\da}\bra{\da}\\
\bra{\da}_2 \rho \ket{\da}_2 &= \fr{1}{2(1+\alpha^{-2})}\ket{\ua}\bra{\ua}\\
\bra{\ua\da}_2 \rho \ket{\ua\da}_2 &= \fr{\alpha^{-2}}{2(1+\alpha^{-2})}\ket{0}\bra{0}\\
\eeq
Therefore,
\beq
\rho_1 &= \fr{1}{2(1+\alpha^{-2})}\qq{\ket{\ua}\bra{\ua}+ \ket{\da}\bra{\da} + \alpha^{-2}\rr{\ket{\ua\da}\bra{\ua\da} + \ket{0}\bra{0}}}\\
       &=\fr{1}{2(1+\alpha^{-2})}\begin{pmatrix}1 & & & \\ & 1 & & \\ & & \alpha^{-2} & \\ & & & \alpha^{-2} \end{pmatrix}
\eeq
The von Neumann entropy is given by \il{S_1 = -\sum_i \lambda_i \log \lambda_i}, where \il{\lambda_i} are the eigenvalues.
\beq
S &= -\fr{1}{2(1+\alpha^{-2})}\times 2 \times \qq{\ln \fr{1}{2(1+\alpha^{-2})} + \alpha^{-2}\ln \fr{\alpha^{-2}}{2(1+\alpha^{-2})}}\\
  &=\fr{1}{(1+\alpha^{-2})}\qq{\ln \rr{2+2\alpha^{-2}} + \alpha^{-2}\ln \rr{2+2\alpha^{2}}}
\eeq
\begin{center}
	\includegraphics*[scale=0.4]{S1.png}
\end{center}
The entropy is maximum (\il{\log 4}) at \il{\fr{U}{4t} = 0} and minimum (\il{\log 2}) at \il{\fr{U}{4t} \ra \infty}. This is expected for the following reason: At \il{U = 0}, the ground state becomes an equal admixture of all four states. If a measurement is performed on site 2, site 1 ends up in any one of the four possible configurations (\il{\ket{0},\ket{\ua},\ket{\da},\ket{\ua\da}}), with equal probability. The fact that the probabilities are equal means the resultant state will be maximally entangled, which leads to the entropy taking the form \il{\log N}, where \il{N} is the dimension of the reduced Hilbert space. Since site 1 can go into four possible states,  the value of \il{N}  is 4, and we get \il{S_1 = \log 4}.\\\\
For \il{U=\infty}, the ground state is just a singlet (\il{\ket{\ua,\da} - \ket{\da,\ua}}), that is, an equal admixture of two states. This means that on measuring site 2, site 1 now has only two options to choose from, \il{\ket{\ua}} or \il{\ket{\da}}, the doublon-holon states are out of the picture. This reduces the dimension of the available Hilbert space to 2, and we get \il{\log 2}.

To find the entanglement content of just the singlet part, I can project out the doublon-holon part from the density matrix.
\beq
\rho_{sg} = \fr{1}{2}\begin{pmatrix}1 & 0 \\ 0 & 1 \end{pmatrix}
\eeq
The entropy will just be \il{\log 2}. Projecting out the singlet part and keeping the doublon-holon part will give the same entropy.
\end{document}
